% Hazi Feladat / Meresi jegyzokony sablon BME MIT
% Keszult: 2012.13.17
% Leiras: Ebbe a fajlba kerul a lenyegi resz, a szoveg. A legfelsobb szintu felsorolas a section (chapter nem hasznalatos).

\section{Bevezetés}
\subsection{Motiváció és célkitűzés}
Az önálló laborom témájának a mutációs tesztelést választottam. Témalaborom alatt találkoztam ezzel a módszerrel, és úgy gondoltam, érdemes vele tovább foglalkozni. Célom volt a félévben elmélyülni ebben a témában, és egy saját mutációs tesztelő eszközt létrehozni. 
\subsection{A mutációs tesztelésről}
Unit tesztelésnél a tesztkészlet minősége többféleképpen mérhető. Általában a programkód tesztek általi lefedettsége a meghatározó szempont, ezt egyszerűe és gyorsan lehet statikus analízis eszközökkel vizsgálni. A lefedettség-mérés viszont csak a lefutott kód hányadát nézi, a program és a tesztek összefüggését nem. Honnan tudjuk, hogy egy, a programban elkövetett hibát biztosan kiszűrnek-e a tesztesetek?

Vizsgáljuk a tesztkészlet és a tesztelendő szoftver kölcsönhatását! Tegyünk apró hibákat mesterségesen az eredeti programkódba, és nézzük meg, hogy a tesztek megtalálják-e ezeket. Az megtalált és az összes hibák arányából készítsünk egy pontszámot. A mutációs tesztelés alapvetően ebből a két lépésből áll.

Mutánsok létrehozásakor mindig csak egy-egy utasítást változtatunk azokban a kódokban, amik a tesztek során végrehajtódnak. Ezeken az utasításokon mutációs operátorokat futtatunk, ezek határozzák meg, hogy egy utasítás milyen másik utasítássá változhat át. Ezekre néhány példa: matematikai (összeadásból kivonás), feltétel negáló (egyenlőből nem egyenlő), visszatérési érték (változóból null). A mutáns programokat ezután lefuttatjuk a tesztkészleten: ha a teszt hibát jelez, akkor jó a teszteset, mivel "megölte" a mutánst. Ha viszont egy mutánsra az összes teszt rendben lefutott, a mutáns "életben maradt", a tesztkészlet nem tudta lebuktatni. Fontos, hogy egy mutánsban egyszerre több operátort nem hajtunk végre: mivel a mutációknak egyenként is felderíthetőknek kell lenniük, több mutáció együttes végrehajtásának nincs információtartalma. 

A mutációs tesztelés egy nagy hátránya, hogy sokkal erőforrásigényesebb, mint egy szimpla lefedettség-mérés. Egy nagyobb program vagy programkönyvtár több ezer mutánssal rendelkezhet, és ezekre a teszteket egyenként végrehajtani igencsak hosszadalmas procedúra. 
\section{Elterjedt mutációs tesztelési eszközök}
A mutációs tesztelés ötlete több, mint 40 éves \cite{major_2011}, ez idő alatt különféle tudású mutációs teszteszközt fejlesztettek ki különböző programnyelvekhez. Elsősorban a Java nyelvhez fejlesztett eszközökre fókuszáltam, mert a JUnit környezettel már sok tapasztalatom volt, és így otthonosabban ki tudtam próbálni ezeket az eszközöket. Két eszközt vizsgáltam meg alaposabban, ezek a Major és a PIT.
\subsection{MAJOR Mutation Framework}
A Major egy fejlett mutációs tesztelőkörnyezet Java platformhoz. Sok szempontból optimalizált, hatékonyan és gyorsan állítja elő és futtatja a mutánsokat. Saját konfigurációs nyelvvel (Major Mutation Language) rendelkezik, amivel a mutációs operátorok részletekbe menően beállíthatóak. Saját beépített Java fordítójának köszönhetően fordítás közben hozza létre a mutációkat, az így előállított mutációk pedig jobban igazodnak a nyelv sajátosságaihoz, és a programozó által potenciálisan elkövethető hibákat fogják szimulálni. A beépített fordító sajnos a hátránya is, mert jelen pillanatban a Major utolsó verziója Java 1.7-es fordítót használ, és Java 1.8-as projektekhez nem használható. Apache Anttel és önmagában is futtatható (az indító shell script némi módosítása után), futása gyors. Az eredmények CSV fileokban állnak elő, amikből részletes statisztika készíthető. \cite{major_manual}
\subsection{PITest}
A PIT egy modernebb eszköz, a kettő közül ennek a használata jobban kézre állt. A Majortól eltérően itt a már lefordított bytekódon történik a mutációs tesztelés, a mutánsok a memóriában jönnek létre és több szálon hajtódnak végre. Mivel a bytekódban jönnek létre a muatációk, előfordulhatnak úgynevezett "junk" mutációk is, amik nem követhetők vissza a forráskódban elkövetett programozási hibákra, de a PIT algoritmusai ezeket képesek kiszűrni. Kényelmesen használható, Maven és Ant pluginokon kívül Eclipse és IntelliJ beépülők is készültek hozzá (habár ez utóbbinál a beállítási lehetőségek igen szegényesek). Mavenes indításnál az eszköz egyszerűbb paraméterei parancssorban adhatók meg~(lásd \ref{lst:pitest_bash}. shell script), a részletes beállítások, mint például a mutációs operátorok, a pom.xml-ben. Ha a program muatációja jellemzően végtelen ciklusokat eredményez, érdemes egy timeout paramétert megadni, ami idő letelte után a PIT leállítja a teszt processzt. \cite{pit_quickstart}

\begin{lstlisting}[frame=single,float=!ht,caption={Példa a PITest paraméterezésére},captionpos=b,label={lst:pitest_bash},language=bash]
mvn test
mvn org.pitest:pitest-maven:mutationCoverage -DmaxDependencyDistance=1 -DtimeoutConstant=250
\end{lstlisting}

Lefuttattam a PIT-et egy olyan projektre, aminek fejlesztésében részt vettem (vasutas játék): a sorlefedettség 18\% (276/1495), a mutációs pontszám pedig 9\% (93/1007) volt. A lefedettség azért lett ilyen alacsony, mert Unit tesztek csak a kontroller és modell osztályokra léteztek, a program jelentős hányadát pedig UI kódok, segédeszközök tették ki. Az eredményt (\ref{lst:pitest_output}. részlet) megvizsgálva látható, hogy mely utasítások hatása nincs vizsgálva a teszt során (hiszen ha lenne, a mutáció során sikertelen lett volna a teszt). A kimutatásból visszakereshető a kódrészlet sorra pontosan, így a hiányosság nagyon gyorsan javítható.

\begin{lstlisting}[frame=single,float=!ht,caption={PIT kimenete},captionpos=b,label={lst:pitest_output},language=bash]
102	1. negated conditional - KILLED
...
116	1. removed call to serpenyoe/model/rails/Tunnel::setActive - KILLED
117	1. removed call to serpenyoe/model/rails/Tunnel::setB - SURVIVED
119	1. removed call to serpenyoe/model/Rail::setA - KILLED
120	1. removed call to serpenyoe/model/Rail::setB - KILLED
\end{lstlisting}

\section{Saját eszköz megtervezése}
Az önálló laborom egyik célja egy saját eszköz megtervezése és implementációja volt. Természetesen a piacon lévő eszközökhöz képest, amiket évek óta fejlesztenek és működésük nagymértékben optimalizált, jelentős egyszerűsítésekkel éltem. A fő szempont a mutációs tesztelés lépéseinek szemléltetése és egy működő prototípus elkészítése volt.

A tevezés során is leginkább az egyszerűségre törekedtem, de ha kettő közül a jobb megoldás nem volt nagyságrendekkel nehezebb, inkább azt választottam. Megfigyeltem a fent említett eszközök működését, és felvázoltam a mutációs tesztelés egyes lépéseit (\ref{fig:steps}. ábra), valamint, hogy milyen lehetőségek vannak azok megvalósítására.

\begin{figure}
\centering
\begin{tikzpicture}[node distance = 1cm, auto]
    % Place nodes
   
    
    \node [block] (coverage) {Code Coverage};
    \node [block_rounded, below=of coverage] (tests) {JUnit tests};
    \node [block, right=of coverage] (mutants) {Mutants};
     \node [block_rounded, above=of coverage] (original) {Original code};
    \node [block, right=of mutants] (evaluate) {Test evaluation};
    \node [block_rounded, above=of evaluate] (operators) {Configuration\\ M.Operators};
    \node [block, right=of evaluate, node distance=3cm] (mutscore) {Mutation score};
    \node [block, below=of mutscore] (reports) {Reports};
    % Draw edges
    \path [line] (original) -- (coverage);
    \path [line] (tests) --  (coverage);
    \path [line] (coverage) -- (mutants);
    \path [line] (original) -| (mutants);
    \path [line] (operators) -| (mutants);
    \path [line] (mutants) -- (evaluate);
    \path [line] (tests) -| (evaluate);
    \path [line] (evaluate) -- (mutscore);
    \path [line] (mutscore) -- (reports);
\end{tikzpicture}

\caption{A mutációs tesztelés lépései} \label{fig:steps}
\end{figure}

\subsection{Forráskód vagy bytekód manipuláció?}
A legfontosabb kérdés, hogy a manipulációt a szoftverfolyamat melyik fázisában hajtjuk végre: fordítás előtt, közben, vagy utána? Mindegyik módszernek megvan a maga előnye és hátránya, és mind sajátos megközelítést kívánnak, ezért ezt a kérdést a tervezés legelején el kellett dönteni.
\subsubsection{Forráskód manipuláció}
Elsőre a forráskód manipuláció egyszerűbbnek és látványosabbnak tűnhet. Itt közvetlenül meg lehet figyelni a mutációs eszköz működését, és a bonyolultabb nyelvi elemek mutációjára is lehetőség van. Viszont itt szükség van java szintaxis értelmező modulra, hogy a módosított forrráskódot előállíthassuk, illetve a fordítás jelentősen időigényesebb lehet, mivel minden egyes mutációnál meg kell ismételni.
\subsubsection{Bytekód manipuláció}
A már lefordított program bytekódját egyszerű módosítani (akár framework segítségével: ASM). Előállhatnak így "junk" mutációk, amik nem vezethetők vissza forráskódban elkövethető hibákra. Előnye, hogy a mutáció egyszerűen elvégezhető a memóriában is, és (egyébként is) kevésbé erőforrásigényes. Hátránya, ha a forrásban látni szeretnénk a mutációkat, azokat vissza kell követni.
\subsubsection{Fordítóba integrált}
Ha a mutációs eszköz a fordítóba van építve, az ötvözni tudja az előző két módszer jó tulajdonságait. Hátránya, hogy viszonylag bonyolult elkészíteni, és meglehetősen rugalmatlan (nem követi a Java verziószámokat, körülményes futtatni), és ilyenkor a mutánsokat fileba kell írni.
\subsubsection{Döntés}
A bytekód manipulációt választottam, mert alapvetően megfelel az igényeimnek, és elég egyszerű ahhoz, hogy az önálló labor alatt megvalósítható legyen.
\subsection{Lefedett kódok azonosítása}
Ahhoz, hogy tudjuk, a mutációkat hova kell beilleszteni a kódba, tudni kell, hogy a vizsgált tesztre mely részek fognak lefutni a programból. A célom az volt, hogy a mutációk csak a tesztosztállyal azonos packageban lévő osztályokon történjenek csak meg, elkerülve azt, hogy külső könyvtárakon és irreleváns kódokon is lefut az eszköz.
\subsubsection{Statikus analízis alapján}
Megvizsgáljuk, hogy a teszt milyen függvényeket hív meg az eredeti programból. Ehhez vagy egy forráskód értelmezőre van szükség, vagy a bytekódból gyűjtjük ki a más osztályokra mutató függvényhívásokat. Ki kell szűrni azokat a programrészeket (libraryket), amik nem tartoznak szorosan a vizsgált programhoz, például a tesztek és az osztály package egyezésének vizsgálatával. Hibája, hogy az osztályok statikus inicializáló függvényét figyelmen kívül hagyja, mivel azokat nem hívja semmi \cite{so_you_want_to}, de ez legtöbb esetben nem okoz problémát.
\subsubsection{Elnevezési konvenciók alapján}
A módszer lényege az, hogy a teszt/tesztkészlet nevéből konvenciók alapján következtetünk, melyik részek/osztályok fognak futni a programból. Legtöbb esetben a tesztosztály neve visszautal arra az osztályra vagy unitra, amihez tartozik, így ez a módszer a külső könyvtárakat, és a mutálni nem kívánt más programkódokat működéséből fakadóan kiszűri. Ez a módszer viszont igen pontatlan lehet, ha nem követjük pontosan az elnevezés szabályait. Előnye viszont, hogy igen egyszerű megvalósítani, csupán String-átalakítások szükségesek hozzá..
\subsubsection{Döntés}
Az ASM képes modellt alkotni a vizsgált bytekódról, tehát arra is megfelelő, hogy a tesztekben a programkód függvényeinek hívásait azonosítsuk vele. Mivel így nem járt többletmunkával, az előnyök miatt a statikus analízist választottam, azzal az egyszerűsítéssel, hogy csak a tesztből közvetlenül hívott függvényeken hajtódik végre mutáció.
\subsection{Mutációs operátorok}
Egy meglévő operátor készletet könnyen ki lehet egészíteni továbbiakkal, erre a projektre a feltétel negáló operátorokat, néhány matematikai, valamint függvényhívás operátort valósítottam meg (a bytekód manipuláció miatt ezek JVM instrukciók, a típusnevek stacken lévő értékeket jelölnek):
\begin{itemize}
  \item \lstinline{ if(int==int) } $ \rightarrow $ \lstinline{ if(int!=int) }
  \item \lstinline{ if(int!=int) } $ \rightarrow $ \lstinline{ if(int==int) }
  \item \lstinline{ if(int>=int) } $ \rightarrow $ \lstinline{ if(int<int) }
  \item \lstinline{ if(int<=int) } $ \rightarrow $ \lstinline{ if(int>int) }
  \item \lstinline{ if(int>int) } $ \rightarrow $ \lstinline{ if(int<=int) }
  \item \lstinline{ if(int<int) } $ \rightarrow $ \lstinline{ if(int>=int) }
  \item \lstinline{ if(int==0) } $ \rightarrow $ \lstinline{ if(int!=0) }
  \item \lstinline{ if(int!=0) } $ \rightarrow $ \lstinline{ if(int==0) }
  \item \lstinline{ if(int>=0) } $ \rightarrow $ \lstinline{ if(int<0) }
  \item \lstinline{ if(int<=0) } $ \rightarrow $ \lstinline{ if(int>0) }
  \item \lstinline{ if(int>0) } $ \rightarrow $ \lstinline{ if(int<=0) }
  \item \lstinline{ if(int<0) } $ \rightarrow $ \lstinline{ if(int>=0) }
    \item \lstinline{ if(Object==null) } $ \rightarrow $ \lstinline{ if(Object!=null) }
    \item \lstinline{ if(Object!=null) } $ \rightarrow $ \lstinline{ if(Object==null) }
        
  \item \lstinline{ float+float } $ \rightarrow $ \lstinline{ float-float }
  \item \lstinline{ double+double } $ \rightarrow $ \lstinline{ double-double }
  \item \lstinline{ long+long } $ \rightarrow $ \lstinline{ long-long }
  \item \lstinline{ int+int } $ \rightarrow $ \lstinline{ int-int }
  \item \lstinline{ float-float } $ \rightarrow $ \lstinline{ float+float }
  \item \lstinline{ double-double } $ \rightarrow $ \lstinline{ double+double }
  \item \lstinline{ long-long } $ \rightarrow $ \lstinline{ long+long }
  \item \lstinline{ int-int } $ \rightarrow $ \lstinline{ int+int }
  
    \item virtuális függvényhívás $ \rightarrow $ nincs művelet
    \item interface függvényhívás $ \rightarrow $ nincs művelet
        
\end{itemize}
Ezek azok az operátorok, amiket viszonylag könnyű volt implementálni, és a legtöbb programra hatékonyan működnek. 
\subsection{Mutációk beillesztése kódba}
Fontos szempont, hogy a mutációkor létrehozott kódok később, a futáskor milyen módon lesznek azonosíthatóak. Többféle módszer létezik a mutációk generálás utáni rendszerezésére, hogy a tesztek futtatásakor kiválaszthatók legyenek.
\subsubsection{Elágazás-struktúra}
Az első esetben egy program jön létre, ami az összes mutációt tartalmazza elágazásos szerkezetben. Futáskor a megfelelő flag beállításával kiválasztható az az egy mutáció, ami le fog futni. A mutáció azonosítóját paraméterként kell átadni a programnak, és gondoskodni róla, hogy az flag formájában eljusson az elágazásig. A legtöbb teszteszköz ezt alkalmazza, mert hatékony, takarékos a tárhellyel. Emiatt, ha a mutációk a memóriában történnek, szinte csak ez jöhet szóba.
\subsubsection{Külön fileba írás}
Minden mutáció egy külön class filet kap. Az egyes mutációkat a filenév azonosítja. Ez a módszer több tárhelyet foglal, de egyszerűbben végrehajtható.
\subsubsection{Döntés}
Végül a fileba írás mellett döntöttem, mivel a JUniton keresztül körülményes a paraméterek átadása. A mutációk átmásolódnak az eredeti class helyére a JUnit futása előtt, a futás után pedig törlődnek.
\subsection{Tesztek futtatása, kiértékelése}
Az eszköz a mutánsokat kiértékelteti a JUnit segítségével. Paraméterként a JUnitnak át kell adni valamilyen módon, hogy mit hajtson végre, a kimenenetét pedig értelmezni kell. Arra jutottam, hogy egy shell program segítségével ez egyszerűen megoldható, hiszen a mutáns a JUnit futása előtt másolással lett kiválasztva, a kimenetet (OK/Failed) pedig a grep segédprogram segítségével egyszerűen lehet értelmezni. A programom keretét ezért egy egyszerű shell script adja, és csak a mutációk előállítását fogja egy Java program végezni.
\subsubsection{Felügyelet a futtatás felett, párhuzamos futtatás}
Egy ciklus mutációjakor (vagy egyébként is) létrejöhet olyan helyzet, hogy a tesztelt program végtelen ciklusba kerül, vagy egyszerűen rendellenesen sokáig tart egy teszteset futása. Szükség lehet egy maximum futási idő meghatározására, ami után a főprogram kilövi a teszt processt. A tervezett operátorkészlet feltétel negáló operátorai sajnos potenciálisan előállíthatnak ilyen helyzetet. A fejlettebb teszteszközök képesek arra is, hogy több szálon, akár több processzormagot használva több mutációt futtatnak egyszerre. A processek párhuzamos kezelése viszont sajnos egy olyan extra funkció, ami nem férne bele az önálló labor kereteibe.
\subsection{Kimenet formátuma}
A kimenet formátumára rengetegféle lehetőség van, az elterjedt eszközök is sokféleképpen oldják meg. Az eredményt legjobban szemléltető módszer HTML formátumban, diagramokat és hiperhivatkozásokat használni. CSV-ben is eltárolhatók az adatok, a későbbi könnyű feldolgozás érdekében. Mivel egyelőre az eredménnyel semmi más tervem nincs, úgy döntöttem, hogy az egyes pontszámokat a konzol kimenetére írom ki.
\section{Saját eszköz fejlesztése}
A félévben fejlesztett szoftverem egy parancssoros eszköz: paraméterként egy Java projekt elérési útját kapja, kimenetként pedig előállítja az osztályok mutációs pontszámát. Az eszköz Maven könyvtárstruktúrát feltételez, a megfelelő mappákban keresi a lefordított program bytekódot és teszt bytekódot. Futtatás előtt ajánlott kiadni a \lstinline{ mvn test } parancsot, hogy a megfelelő állományok előálljanak, valamint hogy meggyőződjünk arról, hogy a tesztkészlet sikerrel lefut, hiszen a mutációs tesztelésnek csak ekkor van értelme.
\subsection{Programstruktúra}
A mutációs tesztelőeszközöm két fő részből áll: egy, a mutációkat előállító Java program, és egy UNIX shellben futtatható keret script. A shell script feladata a tesztek megkeresése, rájuk a mutációkat előállító Java program meghívása, majd a létrejött mutáns osztályokra egyenként a tesztelést elindítani. Ezután a script egybegyűjti az eredményeket, és a mutációs pontszámot osztályokra lebontva mutatja.
\subsection{ASM Framework}
Ahhoz, hogy a Java bytekódhoz hatékonyan hozzáférhessek, az ASM Java Bytecode Manipulation Frameworköt használtam. Ez a könyvtár lehetővé teszi, hogy a kód manuális beolvasása helyett, a kódstruktúrát már feldolgozva kapjuk. A bytekód Visitor elvvel járhetó be, az egyes függvényeket, utasításokat könnyedén lehet egyenként kezelni. Így egyaránt alkalmas a tesztosztályok statikus analízisére és a mutációk végrehajtására is. A Visitor metódusokban eszközölt változások a végén visszaírhatók fileba.

Az osztályok beolvasása és kiírása ClassReader és ClassWriter osztályokkal lehetséges~(\ref{lst:asm1}. kódrészlet). A bejárás során először a ClassVisitor osztály lép működésbe, ez lépked végig az osztály elemein. Szempontunkból a visitMethod függvény érdekes, ez adja tovább a vezérlést a MethodVisitor osztálynak. Ebben az egyes utasítások bejárására különböző paraméterezésű függvények vannak~(\ref{lst:asm2}. kódrészlet), amelyek OpCode paramétere azonosítja az utasításokat (Opcodes.IF\_ICMPEQ, Opcodes.FADD...), és itt le is cserélhetők bármelyik másikra. 


\begin{lstlisting}[frame=single,float=!ht,caption={Java bytekód beolvasása, feldolgozása, kiíratása ASM segítségével},captionpos=b,label={lst:asm1}]
ClassReader reader = new ClassReader(new FileInputStream(...));
ClassWriter writer = new ClassWriter(reader, 0);
ClassVisitor visitor = new ClassVisitor(writer);
reader.accept(visitor, 0);
byte[] newBytecode = writer.toByteArray();
FileOutputStream output = new FileOutputStream(...);
output.write(newBytecode);
output.flush();
output.close();
\end{lstlisting}

\begin{lstlisting}[frame=single,float=!ht,caption={A MethodVisitor általam használt bejáró függvényeinek fejlécei},captionpos=b,label={lst:asm2}]
public void visitJumpInsn(int opcode, Label label);
public void visitIntInsn(int opcode, int operand);
public void visitInsn(int opcode);
\end{lstlisting}

\subsection{A mutánsokat előállító Java program}
A Java programom az egyszerűbb adatátadás érdekében két egymásra épülő feladatot hajt végre: a lefedettség analízist, és a mutációk előállítását. A kódbázist ennek megfelelően két packagere osztottam fel: mutator és testanalyzer, ezek működése független egymástól, csak az adatmodell osztályon osztoznak. Ezen kívül egy harmadik package tartalmazza a kontroller Main függvényt~(\ref{lst:tool_main}. kódrészlet), aminek feladata a két feladat elindítása, köztük az adatáramlás biztosítása, valamint a mutator felkonfigurálása mutációs operátorokkal.

\begin{lstlisting}[frame=single,float=!ht,caption={Main függvény},captionpos=b,label={lst:tool_main}]
    public static void main(String[] args) throws Exception {
        List<MethodId> methods;
        methods = TestClassVisitor.analyzeTest(new File(args[0]), "");

        Mutator mutator = new Mutator();
        mutator.addMutationOperator(Opcodes.IF_ICMPEQ, Opcodes.IF_ICMPNE);
        mutator.addMutationOperator(Opcodes.IF_ICMPNE, Opcodes.IF_ICMPEQ);
        // ...
        mutator.addMutationOperator(Opcodes.IFNONNULL, Opcodes.IFNULL);
        mutator.addMutationOperator(Opcodes.INVOKEVIRTUAL, Opcodes.NOP);
        mutator.addMutationOperator(Opcodes.INVOKEINTERFACE, Opcodes.NOP);
        String classpath=getClassesPath(new File(args[0]));
        for(MethodId method : methods){
            MutationClassVisitor.mutateMethod(method, classpath, mutator);
        }
    }
\end{lstlisting}

A Main függvény parancssori paramétere egy tesztosztály elérési útja, az eszköz ezt fogja átvizsgálni, innen hivatkozott függvényekben a mutációkat végrehajtani, majd a mutáns classokat az eredetik könyvtárába kimenteni.

Mindkét feladathoz az ASM keretrendszerre volt szükség: a statikus analízis és a mutációk előállítása is bytekód szinten valósul meg. A ClassVisitor és MethodVisitor osztályok leszármaztatásával ezekhez tudtam igazítani a kódok bejárását.
\subsubsection{Statikus analízis}
Ez a modul egy listába gyűjti a kapott teszt osztályban hivatkozott függvényeket a tesztelt programból. A visszatérő adatstruktúra minden lényeges információt tartalmaz ahhoz, hogy a Mutator modul a hivatkozott részletet megtalálhassa~(\ref{lst:tool_methodid}. kódrészlet).

\begin{lstlisting}[frame=single,float=!ht,caption={MethodId adatstruktúra},captionpos=b,label={lst:tool_methodid}]
public class MethodId {
    public String owner;
    public String name;
    public String descriptor;
    public boolean isInterface;
    // ...
}
\end{lstlisting}

A modul működése egyszerű: a TestClassVisitor~(\ref{lst:tool_testclassvisitor}. kódrészlet) meghívja az egyes JUnit tesztfüggvényekre a TestMethodVisitort~(\ref{lst:tool_testmethodvisitor}. kódrészlet), amiben a visitMethodInsn keresi a külső (eredeti programra hivatkozó) függvényhívásokat. A megtalált hívásokat visszajuttatja a TestClassVisitornak, mely megvizsgálja, hogy az megfelel-e a feltételeknek: a függvény osztálya azonos packageben legyen, mint a tesztosztály, illetve még ne szerepeljen a listában (mivel a mutációkat tesztosztályonként maximum egyszer szeretnénk csak végrehajtani).

\begin{lstlisting}[frame=single,float=!ht,caption={TestClassVisitor},captionpos=b,label={lst:tool_testclassvisitor}]
public class TestClassVisitor extends ClassVisitor{
    List<MethodId> visitedMethods = new LinkedList<>();
    // ...
    public void pushMethod(MethodId method){
        if(!checkPackage(method.owner))return;
        for(MethodId mid : visitedMethods){
            if(mid.equals(method))return;
        }

        visitedMethods.add(method);
    }
    @Override
    public MethodVisitor visitMethod(int access, String name, String descriptor, 
                                        String signature, String[] exceptions) {
        // ...
        return new TestMethodVisitor(DEFAULT_API, super.visitMethod(access, name, 
                                     descriptor, signature, exceptions),this);
    }
    // ...
}
\end{lstlisting}
\begin{lstlisting}[frame=single,float=!ht,caption={TestMethodVisitor},captionpos=b,label={lst:tool_testmethodvisitor}]
public class TestMethodVisitor extends MethodVisitor {
    // ...
    @Override
    public void visitMethodInsn(int opcode, String owner, String name, 
                                  String descriptor, boolean isInterface) {
        if(opcode== Opcodes.INVOKEVIRTUAL){
            mtv.pushMethod(new MethodId(owner, name, descriptor, isInterface));
        }
        super.visitMethodInsn(opcode, owner, name, descriptor, isInterface);
    }
}
\end{lstlisting}

\subsubsection{Mutator}

A Main függvény a statikus analízis során kapott függvényekre sorban meghívja a mutateMethod függvényt~(\ref{lst:tool_mutatemethod}. kódrészlet). Ennek feladata a kapott függvény többszöri bejárása, egészen addig, amíg az összes lehetséges mutáció nem lett végrehajtva rajta. A mutáció adatait egy Mutation nevű adatstruktúra~(\ref{lst:tool_mutation_class}. kódrészlet) tartalmazza, ami a bejárás során~(\ref{lst:tool_mutationmethodvisitor}. kódrészlet) feltöltődik a megfelelő adatokkal. A Mutator osztály feladata a mutációs operátorok tárolása, valamint egy lista a már elvégzett mutációkról (hogy a bejárások során ugyanaz a mutáns ne jöjjön létre többször).
\begin{lstlisting}[frame=single,float=!ht,caption={Mutate Method},captionpos=b,label={lst:tool_mutatemethod}]
    public static void mutateMethod(MethodId mid, String cp, Mutator mut) throws IOException {
        File originalFile = new File(cp+"/"+mid.owner+".class");
        File renamedOriginalFile = new File(cp+"/"+mid.owner+".class.original");
        originalFile.renameTo(renamedOriginalFile);

        Mutation m;
        do{
            m = mut.createMutation();
            ClassReader cr = new ClassReader(new FileInputStream(renamedOriginalFile));
            ClassWriter cw = new ClassWriter(cr,0);
            MutationClassVisitor cv = new MutationClassVisitor(cw, mid, m);
            cr.accept(cv, 0);
            byte[] newBytecode = cw.toByteArray();
            if(m.label==null) break;
            File mutationFile = new File(cp+"/"+mid.owner+".class."+m.label+".mutation");
            FileOutputStream fos = new FileOutputStream(mutationFile);
            fos.write(newBytecode);
            fos.flush();
            fos.close();
        }while(true);
    }
\end{lstlisting}

Az egyes mutációk egy sorszámmal lettek azonosítva, ami egyértelműen meghatározza a tartalmazó függvényt és a mutáció helyét. Ez az azonosító a Mutation osztályon keresztül jut el a mutateMethod függvényhez, ahol az a filenév részeként használja, és ez kerül bele a Mutator osztály listájába is. A canMutate függvény visszaadja, hogy a mutáció lehetséges-e vagy nem: az alapján, hogy volt-e már a mutáció, illetve az operátorok között szerepel-e. Ezután a mutate függvény visszaadja a megfelelő mutáns Opcodeot, és felveszi a mutációt a listába. Amennyiben az opcode változott, a visit metódusok visszatérte után az ASM keretrendszer végrehajtja a megfelelő változtatást a bytekódon.

\begin{lstlisting}[frame=single,float=!ht,caption={MutationMethodVisitor},captionpos=b,label={lst:tool_mutationmethodvisitor}]
public class MutationMethodVisitor extends MethodVisitor {
    private Mutation mutation;
    private int opCounter=0;
    // ...
    private int mutate(int opcode){
        opCounter++;
        String opLabel = "OP" + opCounter;
        if(mutation.canMutate(opLabel,opcode)){
            int mutated=mutation.mutate(opLabel, opcode);
            return mutated;
        }
        return opcode;
    }
    // ...
    @Override
    public void visitIntInsn(int opcode, int operand) {
        opcode=mutate(opcode);
        super.visitIntInsn(opcode, operand);
    }
}
\end{lstlisting}

\begin{lstlisting}[frame=single,float=!ht,caption={A Mutation osztály},captionpos=b,label={lst:tool_mutation_class}]
public class Mutation {
    Mutator mut;
    String label=null;

    public Mutation(Mutator mut) {
        this.mut = mut;
    }
    public boolean canMutate(String label, int opcode){
        return (!mut.mutatedLabels.contains(label)) && mut.mutationOperators.containsKey(opcode) 
                               && (this.label==null);
    }
    public int mutate(String label, int opcode){
        this.label=label;
        mut.mutatedLabels.add(label);
        return mut.mutationOperators.get(opcode);

    }
}
\end{lstlisting}
A Java program futása során előálltak a mutáns classok, a megfelelő package könyvtárban, futásra készen.
\subsection{Shell program}
A shell script feladata a shellből egyszerűbben elvégezhető feladatok megoldása: a tesztosztályok, majd a mutánsok megtalálása (find), JUnit futtatása és az eredmények értelmezése (java+grep), mutációs pontszám összesítése (wc+grep). Az egyes string műveleteket reguláris kifejezésekkel végeztem.

A JUnit futtatása minden egyes mutációra lezajlik, a megelelő mutáció bemásolódik az eredeti class helyére.  A program futásának végén helyreállnak az előzetesen kimentett eredeti class fileok.

Az eredmények összesítése a projekten belül létrehozott mappában történt: megölt mutáció esetén a megfelelő classhoz tartozó fileba egy 1-es került, életben maradt mutánshoz 0. A végén az 1-esek aránya az összeshez jelentette a mutációs pontszámot.
\begin{lstlisting}[frame=single,float=!ht,caption={Részlet a shell scriptből},captionpos=b,label={lst:tool_shell_junit},language=bash]
  echo "RUNNING $(basename $mut) [$clsname]"
  result=$(java -cp "$BASEDIR/lib/*:$1/target/test-classes:$1/target/classes" \
                                       org.junit.runner.JUnitCore $2 | grep "Tests run\|OK")
  if echo $result | grep "OK"; then
    echo "0" >> "$1/target/arontest-reports/$clsname"
  else
    echo "1" >> "$1/target/arontest-reports/$clsname"
  fi
\end{lstlisting}

\subsection{Kipróbálás}
Az elkészült eszközt először egy nagyon egyszerű projekttel próbáltam, ami egyetlen függvényt tartalmazott. A függvényen belüli összehasonlító elágazásokkal jól lehetett szemléltetni a mutációs operátorok működését. 

\begin{lstlisting}[frame=single,float=!ht,caption={Simple Class},captionpos=b,label={lst:simpleclass}]
public class SimpleClass {
    public int something(int param){
        if(param<1) hey(1);
        if(param>2){
            return 2;
        }else{
            if(param==3)return 4;
        }
        return param;
    }
    void hey(int i){
    }
}
\end{lstlisting}
\begin{lstlisting}[frame=single,float=!ht,caption={Simple Class Test},captionpos=b,label={lst:simpleclasstest}]
public class SimpleClassTest {
    @Test
    public void SomethingGreater(){
        SimpleClass h = new SimpleClass();
        Assert.assertEquals(2,h.something(3));
    }
    @Test
    public void SomethingSmaller(){
        SimpleClass h = new SimpleClass();
        Assert.assertEquals(1,h.something(1));
    }
}
\end{lstlisting}
\begin{lstlisting}[frame=single,float=!ht,caption={Az eszköz kimenete erre a programra},captionpos=b,label={lst:simpleclass_run},language=bash]
WITHOUT MUTATION
OK (2 tests)
ANALYZING TEST [...] testproj/SimpleClassTest.class
METHOD testproj/SimpleClass/something
CREATED MUTATION OP7131091982 (IF_ICMPGE->IF_ICMPLT)
WRITTEN SimpleClass.class.OP7131091982.mutation
CREATED MUTATION OP7131091984 (INVOKEVIRTUAL->NOP)
WRITTEN SimpleClass.class.OP7131091984.mutation
CREATED MUTATION OP7131091986 (IF_ICMPLE->IF_ICMPGT)
WRITTEN SimpleClass.class.OP7131091986.mutation
CREATED MUTATION OP71310919810 (IF_ICMPNE->IF_ICMPEQ)
WRITTEN SimpleClass.class.OP71310919810.mutation
METHOD testproj/SimpleClass/something
RUNNING SimpleClass.class.OP71310919810.mutation [testproj.SimpleClass]
Killed (Tests run: 2,  Failures: 1)
RUNNING SimpleClass.class.OP7131091982.mutation [testproj.SimpleClass]
OK (2 tests)
Survived (OK (2 tests))
RUNNING SimpleClass.class.OP7131091984.mutation [testproj.SimpleClass]
OK (2 tests)
Survived (OK (2 tests))
RUNNING SimpleClass.class.OP7131091986.mutation [testproj.SimpleClass]
Killed (Tests run: 2,  Failures: 2)
testproj.SimpleClass: 2/4
\end{lstlisting}
Működés közben az eszköz kiírja a standard outputra az éppen elvégzett feladatot, így könnyen nyomon lehet követni a működését. Az utolsó sorban az eszköz kiírja az egyes osztályokra adott mutációs pontot: ebben az esetben az egyetlen osztály négy mutációjából kettőt sikerült megölni. A kimenetből látható, hogy az eszköz a várt módon működik: a visszaadott értékre hatással lévő elágazások mutációját lebuktatja a teszten belüli assert. A függvény elején lévő elágazást és a függvényhívást viszont a tesztkészlet nem tudta vizsgálni, így a belőlük létrejött mutációk életben maradtak. 
\subsubsection{Hiányosságok és nehézségek}
Függvényhívások törlésekor a programom nem tudta lekövetni a stack méretének változtatását. Mivel a program csak magát a hívást detektálta, a függvényparamétereket a stackbe helyező és a visszatérési értéket onnan kivevő utasítások a bytekódban maradtak. A programom arra az esetre jól működhet, amikor a paraméterek mérete és a visszatérési értékek mérete megegyezik (habár ilyenkor sem mindig), egyéb esetben viszont mindig hibát jelez (és ekkor hibásan a megölt mutánsok közé számolja ezt az esetet).

A mutációs teszteszközöm a tesztosztályokra egymástól függetlenül hajtja végre a mutációkat, valamint futtatja azokat, így több tesztosztály esetén elképzelhető, hogy ugyanaz a mutáció többször létrejön, és akár más-más eredményt adnak. Azonban mivel csak a tesztosztályhoz szorosan kapcsolódó (vele egy packageben lévő, egy hívás mélységben lévő) kódra hajt végre mutációkat, illetve általában egy tesztosztály egy unitot tesztel, ez a legtöbb esetben nem probléma.
\subsubsection{Tesztelés egy nagy projekten}
A szoftveremet egy nagy projektre is lefuttattam, erre a célra a docx4j (plutext/docx4j) nevű projekt megfelelt, mivel Java nyelvű, kevés külső függőséget alkalmaz, sok teszttel rendelkezik, és Mavenes projekt. Éltem azzal az egyszerűsítéssel, hogy a tesztosztályokból csak hármat hagytam meg, a gyors futás érdekében. 

A projektet lefuttattam a PITesttel is, és törekedtem arra, hogy minél hasonlóbb paraméterezéssel fusson a két program. A mutációs operátorok közül a feltétel negáló és matematikai operátorokat alkalmaztam. A kimeneten~(\ref{resultstable}. táblázat) látszik, hogy sok esetben hasonló eredményre jutott a két eszköz, habár az utolsó osztálynál a saját eszköz furcsa módon az összes mutációt lebuktatta (lehetséges, hogy a mutációk futtatásakor hiba történt).

\begin{table}[]
\centering
\def\arraystretch{1.5}
\begin{tabular}{|l|l|l|}
\hline
                                        & \textbf{PITest} & \textbf{Saját eszköz} \\ \hline
\textbf{org.docx4j.toc.TocGenerator}    & 12/41           & 9/46                  \\ \hline
\textbf{org.docx4j.wml.SdtBlock}        & 0/8             & 0/20                  \\ \hline
\textbf{org.docx4j.wml.SdtContentBlock} & 1/3             & 4/4                   \\ \hline
\textbf{org.docx4j.wml.SdtPr}           & 4/19            & 24/24                 \\ \hline
\end{tabular}
\caption{A docx4j mutációs tesztelésének eredménye}
\label{resultstable}
\end{table}

\section{Végszó}
A félévben sikerült mélyebben megismerkednem a mutációs tesztelés lehetőségeivel, technikáival, számba vettem az ehhez szükséges lépéseket, majd kifejlesztettem egy saját eszközt, ami során kitapasztaltam a bytekód manipuláció működését. Az elkészült eszköz kicsi és demonstrációs célra alkalmas, de teljesítményben és lehetőségekben nem tud versenybe szállni a piacon lévő eszközökkel.

Úgy érzem, ezt a témakört a lehetőségekhez mérten kimerítettem, a következőkben a tesztelés egy más témakörével szeretnék majd foglalkozni.

\begin{thebibliography}{9}

\bibitem{major_2011} 
Rene Just, Franz Schweiggert, and Gregory M. Kapfhammer. 
\textit{MAJOR: An Efficient and Extensible Tool for Mutation Analysis in a Java Compiler}. 
\\\texttt{http://people.cs.umass.edu/\textasciitilde rjust/publ/major\textunderscore compiler\textunderscore ase\textunderscore 2011.pdf}
\bibitem{major_manual} 
\textit{The Major Mutation Framework Manual}, version 1.3.2 
\\\texttt{http://mutation-testing.org/doc/major.pdf}
\bibitem{pit_quickstart} 
\textit{PIT Quick Start}
\\\texttt{http://pitest.org/quickstart/}
\bibitem{so_you_want_to} 
Henry Coles. 
\textit{So you want to build a mutation testing system}. 
\\\texttt{https://github.com/hcoles/pitest/}



\end{thebibliography}